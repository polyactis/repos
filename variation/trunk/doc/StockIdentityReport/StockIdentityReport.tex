\documentclass[a4paper,10pt]{article}
\usepackage[pdftex]{color,graphicx}
\usepackage[pdftex, bookmarks, colorlinks=false]{hyperref}
\usepackage{supertabular}
\usepackage{longtable}
\usepackage{tabularx}

%\pdfpagewidth and \pdfpageheight is only good for pdflatex
%\pdfpagewidth 13in
%\pdfpageheight 22in

\addtolength{\textheight}{5cm}
\addtolength{\textwidth}{5cm}
\addtolength{\hoffset}{-3cm}
\addtolength{\voffset}{-3cm}

%opening
\title{Report on Identity Ecotypes in Database Stock}
\author{Yu Huang}

\begin{document}

\maketitle

\begin{abstract}
this is an expansion of section \emph{identity strains across globe} in outcrossing.pdf
\end{abstract}

\tableofcontents


\section{Identity Cliques}
Here is to see how identity strain pairs are distributed across globe.  the criteria to define identity is this. If one of the two calls is NA, it's deemed as identical too. Hence, the transivity of identity relationship is not guaranteed. For example, given strain 1 and 2 are identical and strain 1 and 3 are also identical, strain 1 and 3 might not be identical.

Think of identity relationship as a graph, which can be partitioned into connected components.

Any two nodes inside a connected component might not be directly connected to each other. They however could be reached via some intermediate nodes.

Each connected component could loosely correspond to a distinct haplotype. 'loosely' because its transivity is not guaranteed and there're still some strains who are not identical to each other. That's why we need cliques.

Any two nodes inside a clique are directly connected to each other. Each clique could correspond to a unique haplotype. Every clique is got by breaking a component. The last section tells you which component each clique is from.

All identity cliques are classified based on continent. Very few cliques are across continents. Even within one continent, most cliques reside in one country.

No figure/map here. all the figure references don't work. there're too many figures to be included in latex.


\begin{figure}[H]
\includegraphics[width=0.5\textwidth]{figures/cluster_plot_AtMSQTsnp_2.png}
\caption{cluster plot for AtMSQTsnp 2.} \label{flAtMSQTsnp2}
\end{figure}

\begin{figure}[H]
\includegraphics[width=0.5\textwidth]{figures/cluster_plot_AtMSQTsnp_4.png}
\caption{cluster plot for AtMSQTsnp 4.} \label{flAtMSQTsnp4}
\end{figure}

\begin{figure}[H]
\includegraphics[width=0.5\textwidth]{figures/cluster_plot_AtMSQTsnp_8.png}
\caption{cluster plot for AtMSQTsnp 8.} \label{flAtMSQTsnp8}
\end{figure}

\begin{figure}[H]
\includegraphics[width=0.5\textwidth]{figures/cluster_plot_AtMSQTsnp_9.png}
\caption{cluster plot for AtMSQTsnp 9.} \label{flAtMSQTsnp9}
\end{figure}

\begin{figure}[H]
\includegraphics[width=0.5\textwidth]{figures/cluster_plot_AtMSQTsnp_10.png}
\caption{cluster plot for AtMSQTsnp 10.} \label{flAtMSQTsnp10}
\end{figure}

\begin{figure}[H]
\includegraphics[width=0.5\textwidth]{figures/cluster_plot_AtMSQTsnp_11.png}
\caption{cluster plot for AtMSQTsnp 11.} \label{flAtMSQTsnp11}
\end{figure}

\begin{figure}[H]
\includegraphics[width=0.5\textwidth]{figures/cluster_plot_AtMSQTsnp_12.png}
\caption{cluster plot for AtMSQTsnp 12.} \label{flAtMSQTsnp12}
\end{figure}

\begin{figure}[H]
\includegraphics[width=0.5\textwidth]{figures/cluster_plot_AtMSQTsnp_14.png}
\caption{cluster plot for AtMSQTsnp 14.} \label{flAtMSQTsnp14}
\end{figure}

\begin{figure}[H]
\includegraphics[width=0.5\textwidth]{figures/cluster_plot_AtMSQTsnp_15.png}
\caption{cluster plot for AtMSQTsnp 15.} \label{flAtMSQTsnp15}
\end{figure}

\begin{figure}[H]
\includegraphics[width=0.5\textwidth]{figures/cluster_plot_AtMSQTsnp_18.png}
\caption{cluster plot for AtMSQTsnp 18.} \label{flAtMSQTsnp18}
\end{figure}

\begin{figure}[H]
\includegraphics[width=0.5\textwidth]{figures/cluster_plot_AtMSQTsnp_21.png}
\caption{cluster plot for AtMSQTsnp 21.} \label{flAtMSQTsnp21}
\end{figure}

\begin{figure}[H]
\includegraphics[width=0.5\textwidth]{figures/cluster_plot_AtMSQTsnp_22.png}
\caption{cluster plot for AtMSQTsnp 22.} \label{flAtMSQTsnp22}
\end{figure}

\begin{figure}[H]
\includegraphics[width=0.5\textwidth]{figures/cluster_plot_AtMSQTsnp_27.png}
\caption{cluster plot for AtMSQTsnp 27.} \label{flAtMSQTsnp27}
\end{figure}

\begin{figure}[H]
\includegraphics[width=0.5\textwidth]{figures/cluster_plot_AtMSQTsnp_29.png}
\caption{cluster plot for AtMSQTsnp 29.} \label{flAtMSQTsnp29}
\end{figure}

\begin{figure}[H]
\includegraphics[width=0.5\textwidth]{figures/cluster_plot_AtMSQTsnp_30.png}
\caption{cluster plot for AtMSQTsnp 30.} \label{flAtMSQTsnp30}
\end{figure}

\begin{figure}[H]
\includegraphics[width=0.5\textwidth]{figures/cluster_plot_AtMSQTsnp_31.png}
\caption{cluster plot for AtMSQTsnp 31.} \label{flAtMSQTsnp31}
\end{figure}

\begin{figure}[H]
\includegraphics[width=0.5\textwidth]{figures/cluster_plot_AtMSQTsnp_33.png}
\caption{cluster plot for AtMSQTsnp 33.} \label{flAtMSQTsnp33}
\end{figure}

\begin{figure}[H]
\includegraphics[width=0.5\textwidth]{figures/cluster_plot_AtMSQTsnp_38.png}
\caption{cluster plot for AtMSQTsnp 38.} \label{flAtMSQTsnp38}
\end{figure}

\begin{figure}[H]
\includegraphics[width=0.5\textwidth]{figures/cluster_plot_AtMSQTsnp_40.png}
\caption{cluster plot for AtMSQTsnp 40.} \label{flAtMSQTsnp40}
\end{figure}

\begin{figure}[H]
\includegraphics[width=0.5\textwidth]{figures/cluster_plot_AtMSQTsnp_41.png}
\caption{cluster plot for AtMSQTsnp 41.} \label{flAtMSQTsnp41}
\end{figure}

\begin{figure}[H]
\includegraphics[width=0.5\textwidth]{figures/cluster_plot_AtMSQTsnp_47.png}
\caption{cluster plot for AtMSQTsnp 47.} \label{flAtMSQTsnp47}
\end{figure}

\begin{figure}[H]
\includegraphics[width=0.5\textwidth]{figures/cluster_plot_AtMSQTsnp_48.png}
\caption{cluster plot for AtMSQTsnp 48.} \label{flAtMSQTsnp48}
\end{figure}

\begin{figure}[H]
\includegraphics[width=0.5\textwidth]{figures/cluster_plot_AtMSQTsnp_49.png}
\caption{cluster plot for AtMSQTsnp 49.} \label{flAtMSQTsnp49}
\end{figure}

\begin{figure}[H]
\includegraphics[width=0.5\textwidth]{figures/cluster_plot_AtMSQTsnp_53.png}
\caption{cluster plot for AtMSQTsnp 53.} \label{flAtMSQTsnp53}
\end{figure}

\begin{figure}[H]
\includegraphics[width=0.5\textwidth]{figures/cluster_plot_AtMSQTsnp_54.png}
\caption{cluster plot for AtMSQTsnp 54.} \label{flAtMSQTsnp54}
\end{figure}

\begin{figure}[H]
\includegraphics[width=0.5\textwidth]{figures/cluster_plot_AtMSQTsnp_57.png}
\caption{cluster plot for AtMSQTsnp 57.} \label{flAtMSQTsnp57}
\end{figure}

\begin{figure}[H]
\includegraphics[width=0.5\textwidth]{figures/cluster_plot_AtMSQTsnp_58.png}
\caption{cluster plot for AtMSQTsnp 58.} \label{flAtMSQTsnp58}
\end{figure}

\begin{figure}[H]
\includegraphics[width=0.5\textwidth]{figures/cluster_plot_AtMSQTsnp_60.png}
\caption{cluster plot for AtMSQTsnp 60.} \label{flAtMSQTsnp60}
\end{figure}

\begin{figure}[H]
\includegraphics[width=0.5\textwidth]{figures/cluster_plot_AtMSQTsnp_61.png}
\caption{cluster plot for AtMSQTsnp 61.} \label{flAtMSQTsnp61}
\end{figure}

\begin{figure}[H]
\includegraphics[width=0.5\textwidth]{figures/cluster_plot_AtMSQTsnp_62.png}
\caption{cluster plot for AtMSQTsnp 62.} \label{flAtMSQTsnp62}
\end{figure}

\begin{figure}[H]
\includegraphics[width=0.5\textwidth]{figures/cluster_plot_AtMSQTsnp_63.png}
\caption{cluster plot for AtMSQTsnp 63.} \label{flAtMSQTsnp63}
\end{figure}

\begin{figure}[H]
\includegraphics[width=0.5\textwidth]{figures/cluster_plot_AtMSQTsnp_65.png}
\caption{cluster plot for AtMSQTsnp 65.} \label{flAtMSQTsnp65}
\end{figure}

\begin{figure}[H]
\includegraphics[width=0.5\textwidth]{figures/cluster_plot_AtMSQTsnp_67.png}
\caption{cluster plot for AtMSQTsnp 67.} \label{flAtMSQTsnp67}
\end{figure}

\begin{figure}[H]
\includegraphics[width=0.5\textwidth]{figures/cluster_plot_AtMSQTsnp_69.png}
\caption{cluster plot for AtMSQTsnp 69.} \label{flAtMSQTsnp69}
\end{figure}

\begin{figure}[H]
\includegraphics[width=0.5\textwidth]{figures/cluster_plot_AtMSQTsnp_73.png}
\caption{cluster plot for AtMSQTsnp 73.} \label{flAtMSQTsnp73}
\end{figure}

\begin{figure}[H]
\includegraphics[width=0.5\textwidth]{figures/cluster_plot_AtMSQTsnp_76.png}
\caption{cluster plot for AtMSQTsnp 76.} \label{flAtMSQTsnp76}
\end{figure}

\begin{figure}[H]
\includegraphics[width=0.5\textwidth]{figures/cluster_plot_AtMSQTsnp_85.png}
\caption{cluster plot for AtMSQTsnp 85.} \label{flAtMSQTsnp85}
\end{figure}

\begin{figure}[H]
\includegraphics[width=0.5\textwidth]{figures/cluster_plot_AtMSQTsnp_87.png}
\caption{cluster plot for AtMSQTsnp 87.} \label{flAtMSQTsnp87}
\end{figure}

\begin{figure}[H]
\includegraphics[width=0.5\textwidth]{figures/cluster_plot_AtMSQTsnp_88.png}
\caption{cluster plot for AtMSQTsnp 88.} \label{flAtMSQTsnp88}
\end{figure}

\begin{figure}[H]
\includegraphics[width=0.5\textwidth]{figures/cluster_plot_AtMSQTsnp_90.png}
\caption{cluster plot for AtMSQTsnp 90.} \label{flAtMSQTsnp90}
\end{figure}

\begin{figure}[H]
\includegraphics[width=0.5\textwidth]{figures/cluster_plot_AtMSQTsnp_91.png}
\caption{cluster plot for AtMSQTsnp 91.} \label{flAtMSQTsnp91}
\end{figure}

\begin{figure}[H]
\includegraphics[width=0.5\textwidth]{figures/cluster_plot_AtMSQTsnp_92.png}
\caption{cluster plot for AtMSQTsnp 92.} \label{flAtMSQTsnp92}
\end{figure}

\begin{figure}[H]
\includegraphics[width=0.5\textwidth]{figures/cluster_plot_AtMSQTsnp_97.png}
\caption{cluster plot for AtMSQTsnp 97.} \label{flAtMSQTsnp97}
\end{figure}

\begin{figure}[H]
\includegraphics[width=0.5\textwidth]{figures/cluster_plot_AtMSQTsnp_100.png}
\caption{cluster plot for AtMSQTsnp 100.} \label{flAtMSQTsnp100}
\end{figure}

\begin{figure}[H]
\includegraphics[width=0.5\textwidth]{figures/cluster_plot_AtMSQTsnp_101.png}
\caption{cluster plot for AtMSQTsnp 101.} \label{flAtMSQTsnp101}
\end{figure}

\begin{figure}[H]
\includegraphics[width=0.5\textwidth]{figures/cluster_plot_AtMSQTsnp_104.png}
\caption{cluster plot for AtMSQTsnp 104.} \label{flAtMSQTsnp104}
\end{figure}

\begin{figure}[H]
\includegraphics[width=0.5\textwidth]{figures/cluster_plot_AtMSQTsnp_108.png}
\caption{cluster plot for AtMSQTsnp 108.} \label{flAtMSQTsnp108}
\end{figure}

\begin{figure}[H]
\includegraphics[width=0.5\textwidth]{figures/cluster_plot_AtMSQTsnp_114.png}
\caption{cluster plot for AtMSQTsnp 114.} \label{flAtMSQTsnp114}
\end{figure}

\begin{figure}[H]
\includegraphics[width=0.5\textwidth]{figures/cluster_plot_AtMSQTsnp_118.png}
\caption{cluster plot for AtMSQTsnp 118.} \label{flAtMSQTsnp118}
\end{figure}

\begin{figure}[H]
\includegraphics[width=0.5\textwidth]{figures/cluster_plot_AtMSQTsnp_123.png}
\caption{cluster plot for AtMSQTsnp 123.} \label{flAtMSQTsnp123}
\end{figure}

\begin{figure}[H]
\includegraphics[width=0.5\textwidth]{figures/cluster_plot_AtMSQTsnp_126.png}
\caption{cluster plot for AtMSQTsnp 126.} \label{flAtMSQTsnp126}
\end{figure}

\begin{figure}[H]
\includegraphics[width=0.5\textwidth]{figures/cluster_plot_AtMSQTsnp_128.png}
\caption{cluster plot for AtMSQTsnp 128.} \label{flAtMSQTsnp128}
\end{figure}

\begin{figure}[H]
\includegraphics[width=0.5\textwidth]{figures/cluster_plot_AtMSQTsnp_129.png}
\caption{cluster plot for AtMSQTsnp 129.} \label{flAtMSQTsnp129}
\end{figure}

\begin{figure}[H]
\includegraphics[width=0.5\textwidth]{figures/cluster_plot_AtMSQTsnp_130.png}
\caption{cluster plot for AtMSQTsnp 130.} \label{flAtMSQTsnp130}
\end{figure}

\begin{figure}[H]
\includegraphics[width=0.5\textwidth]{figures/cluster_plot_AtMSQTsnp_132.png}
\caption{cluster plot for AtMSQTsnp 132.} \label{flAtMSQTsnp132}
\end{figure}

\begin{figure}[H]
\includegraphics[width=0.5\textwidth]{figures/cluster_plot_AtMSQTsnp_138.png}
\caption{cluster plot for AtMSQTsnp 138.} \label{flAtMSQTsnp138}
\end{figure}

\begin{figure}[H]
\includegraphics[width=0.5\textwidth]{figures/cluster_plot_AtMSQTsnp_140.png}
\caption{cluster plot for AtMSQTsnp 140.} \label{flAtMSQTsnp140}
\end{figure}

\begin{figure}[H]
\includegraphics[width=0.5\textwidth]{figures/cluster_plot_AtMSQTsnp_142.png}
\caption{cluster plot for AtMSQTsnp 142.} \label{flAtMSQTsnp142}
\end{figure}

\begin{figure}[H]
\includegraphics[width=0.5\textwidth]{figures/cluster_plot_AtMSQTsnp_143.png}
\caption{cluster plot for AtMSQTsnp 143.} \label{flAtMSQTsnp143}
\end{figure}

\begin{figure}[H]
\includegraphics[width=0.5\textwidth]{figures/cluster_plot_AtMSQTsnp_145.png}
\caption{cluster plot for AtMSQTsnp 145.} \label{flAtMSQTsnp145}
\end{figure}

\begin{figure}[H]
\includegraphics[width=0.5\textwidth]{figures/cluster_plot_AtMSQTsnp_146.png}
\caption{cluster plot for AtMSQTsnp 146.} \label{flAtMSQTsnp146}
\end{figure}

\begin{figure}[H]
\includegraphics[width=0.5\textwidth]{figures/cluster_plot_AtMSQTsnp_155.png}
\caption{cluster plot for AtMSQTsnp 155.} \label{flAtMSQTsnp155}
\end{figure}

\begin{figure}[H]
\includegraphics[width=0.5\textwidth]{figures/cluster_plot_AtMSQTsnp_156.png}
\caption{cluster plot for AtMSQTsnp 156.} \label{flAtMSQTsnp156}
\end{figure}

\begin{figure}[H]
\includegraphics[width=0.5\textwidth]{figures/cluster_plot_AtMSQTsnp_159.png}
\caption{cluster plot for AtMSQTsnp 159.} \label{flAtMSQTsnp159}
\end{figure}

\begin{figure}[H]
\includegraphics[width=0.5\textwidth]{figures/cluster_plot_AtMSQTsnp_164.png}
\caption{cluster plot for AtMSQTsnp 164.} \label{flAtMSQTsnp164}
\end{figure}

\begin{figure}[H]
\includegraphics[width=0.5\textwidth]{figures/cluster_plot_AtMSQTsnp_169.png}
\caption{cluster plot for AtMSQTsnp 169.} \label{flAtMSQTsnp169}
\end{figure}

\begin{figure}[H]
\includegraphics[width=0.5\textwidth]{figures/cluster_plot_AtMSQTsnp_170.png}
\caption{cluster plot for AtMSQTsnp 170.} \label{flAtMSQTsnp170}
\end{figure}

\begin{figure}[H]
\includegraphics[width=0.5\textwidth]{figures/cluster_plot_AtMSQTsnp_173.png}
\caption{cluster plot for AtMSQTsnp 173.} \label{flAtMSQTsnp173}
\end{figure}

\begin{figure}[H]
\includegraphics[width=0.5\textwidth]{figures/cluster_plot_AtMSQTsnp_174.png}
\caption{cluster plot for AtMSQTsnp 174.} \label{flAtMSQTsnp174}
\end{figure}

\begin{figure}[H]
\includegraphics[width=0.5\textwidth]{figures/cluster_plot_AtMSQTsnp_177.png}
\caption{cluster plot for AtMSQTsnp 177.} \label{flAtMSQTsnp177}
\end{figure}

\begin{figure}[H]
\includegraphics[width=0.5\textwidth]{figures/cluster_plot_AtMSQTsnp_184.png}
\caption{cluster plot for AtMSQTsnp 184.} \label{flAtMSQTsnp184}
\end{figure}

\begin{figure}[H]
\includegraphics[width=0.5\textwidth]{figures/cluster_plot_AtMSQTsnp_186.png}
\caption{cluster plot for AtMSQTsnp 186.} \label{flAtMSQTsnp186}
\end{figure}

\begin{figure}[H]
\includegraphics[width=0.5\textwidth]{figures/cluster_plot_AtMSQTsnp_187.png}
\caption{cluster plot for AtMSQTsnp 187.} \label{flAtMSQTsnp187}
\end{figure}

\begin{figure}[H]
\includegraphics[width=0.5\textwidth]{figures/cluster_plot_AtMSQTsnp_188.png}
\caption{cluster plot for AtMSQTsnp 188.} \label{flAtMSQTsnp188}
\end{figure}

\begin{figure}[H]
\includegraphics[width=0.5\textwidth]{figures/cluster_plot_AtMSQTsnp_189.png}
\caption{cluster plot for AtMSQTsnp 189.} \label{flAtMSQTsnp189}
\end{figure}

\begin{figure}[H]
\includegraphics[width=0.5\textwidth]{figures/cluster_plot_AtMSQTsnp_191.png}
\caption{cluster plot for AtMSQTsnp 191.} \label{flAtMSQTsnp191}
\end{figure}

\begin{figure}[H]
\includegraphics[width=0.5\textwidth]{figures/cluster_plot_AtMSQTsnp_194.png}
\caption{cluster plot for AtMSQTsnp 194.} \label{flAtMSQTsnp194}
\end{figure}

\begin{figure}[H]
\includegraphics[width=0.5\textwidth]{figures/cluster_plot_AtMSQTsnp_197.png}
\caption{cluster plot for AtMSQTsnp 197.} \label{flAtMSQTsnp197}
\end{figure}

\begin{figure}[H]
\includegraphics[width=0.5\textwidth]{figures/cluster_plot_AtMSQTsnp_198.png}
\caption{cluster plot for AtMSQTsnp 198.} \label{flAtMSQTsnp198}
\end{figure}

\begin{figure}[H]
\includegraphics[width=0.5\textwidth]{figures/cluster_plot_AtMSQTsnp_203.png}
\caption{cluster plot for AtMSQTsnp 203.} \label{flAtMSQTsnp203}
\end{figure}

\begin{figure}[H]
\includegraphics[width=0.5\textwidth]{figures/cluster_plot_AtMSQTsnp_205.png}
\caption{cluster plot for AtMSQTsnp 205.} \label{flAtMSQTsnp205}
\end{figure}

\begin{figure}[H]
\includegraphics[width=0.5\textwidth]{figures/cluster_plot_AtMSQTsnp_214.png}
\caption{cluster plot for AtMSQTsnp 214.} \label{flAtMSQTsnp214}
\end{figure}

\begin{figure}[H]
\includegraphics[width=0.5\textwidth]{figures/cluster_plot_AtMSQTsnp_220.png}
\caption{cluster plot for AtMSQTsnp 220.} \label{flAtMSQTsnp220}
\end{figure}

\begin{figure}[H]
\includegraphics[width=0.5\textwidth]{figures/cluster_plot_AtMSQTsnp_222.png}
\caption{cluster plot for AtMSQTsnp 222.} \label{flAtMSQTsnp222}
\end{figure}

\begin{figure}[H]
\includegraphics[width=0.5\textwidth]{figures/cluster_plot_AtMSQTsnp_223.png}
\caption{cluster plot for AtMSQTsnp 223.} \label{flAtMSQTsnp223}
\end{figure}

\begin{figure}[H]
\includegraphics[width=0.5\textwidth]{figures/cluster_plot_AtMSQTsnp_231.png}
\caption{cluster plot for AtMSQTsnp 231.} \label{flAtMSQTsnp231}
\end{figure}

\begin{figure}[H]
\includegraphics[width=0.5\textwidth]{figures/cluster_plot_AtMSQTsnp_232.png}
\caption{cluster plot for AtMSQTsnp 232.} \label{flAtMSQTsnp232}
\end{figure}

\begin{figure}[H]
\includegraphics[width=0.5\textwidth]{figures/cluster_plot_AtMSQTsnp_235.png}
\caption{cluster plot for AtMSQTsnp 235.} \label{flAtMSQTsnp235}
\end{figure}

\begin{figure}[H]
\includegraphics[width=0.5\textwidth]{figures/cluster_plot_AtMSQTsnp_237.png}
\caption{cluster plot for AtMSQTsnp 237.} \label{flAtMSQTsnp237}
\end{figure}

\begin{figure}[H]
\includegraphics[width=0.5\textwidth]{figures/cluster_plot_AtMSQTsnp_242.png}
\caption{cluster plot for AtMSQTsnp 242.} \label{flAtMSQTsnp242}
\end{figure}

\begin{figure}[H]
\includegraphics[width=0.5\textwidth]{figures/cluster_plot_AtMSQTsnp_244.png}
\caption{cluster plot for AtMSQTsnp 244.} \label{flAtMSQTsnp244}
\end{figure}

\begin{figure}[H]
\includegraphics[width=0.5\textwidth]{figures/cluster_plot_AtMSQTsnp_249.png}
\caption{cluster plot for AtMSQTsnp 249.} \label{flAtMSQTsnp249}
\end{figure}

\begin{figure}[H]
\includegraphics[width=0.5\textwidth]{figures/cluster_plot_AtMSQTsnp_254.png}
\caption{cluster plot for AtMSQTsnp 254.} \label{flAtMSQTsnp254}
\end{figure}

\begin{figure}[H]
\includegraphics[width=0.5\textwidth]{figures/cluster_plot_AtMSQTsnp_260.png}
\caption{cluster plot for AtMSQTsnp 260.} \label{flAtMSQTsnp260}
\end{figure}

\begin{figure}[H]
\includegraphics[width=0.5\textwidth]{figures/cluster_plot_AtMSQTsnp_263.png}
\caption{cluster plot for AtMSQTsnp 263.} \label{flAtMSQTsnp263}
\end{figure}

\begin{figure}[H]
\includegraphics[width=0.5\textwidth]{figures/cluster_plot_AtMSQTsnp_266.png}
\caption{cluster plot for AtMSQTsnp 266.} \label{flAtMSQTsnp266}
\end{figure}

\begin{figure}[H]
\includegraphics[width=0.5\textwidth]{figures/cluster_plot_AtMSQTsnp_267.png}
\caption{cluster plot for AtMSQTsnp 267.} \label{flAtMSQTsnp267}
\end{figure}

\begin{figure}[H]
\includegraphics[width=0.5\textwidth]{figures/cluster_plot_AtMSQTsnp_274.png}
\caption{cluster plot for AtMSQTsnp 274.} \label{flAtMSQTsnp274}
\end{figure}

\begin{figure}[H]
\includegraphics[width=0.5\textwidth]{figures/cluster_plot_AtMSQTsnp_278.png}
\caption{cluster plot for AtMSQTsnp 278.} \label{flAtMSQTsnp278}
\end{figure}

\begin{figure}[H]
\includegraphics[width=0.5\textwidth]{figures/cluster_plot_AtMSQTsnp_279.png}
\caption{cluster plot for AtMSQTsnp 279.} \label{flAtMSQTsnp279}
\end{figure}

\begin{figure}[H]
\includegraphics[width=0.5\textwidth]{figures/cluster_plot_AtMSQTsnp_281.png}
\caption{cluster plot for AtMSQTsnp 281.} \label{flAtMSQTsnp281}
\end{figure}

\begin{figure}[H]
\includegraphics[width=0.5\textwidth]{figures/cluster_plot_AtMSQTsnp_282.png}
\caption{cluster plot for AtMSQTsnp 282.} \label{flAtMSQTsnp282}
\end{figure}

\begin{figure}[H]
\includegraphics[width=0.5\textwidth]{figures/cluster_plot_AtMSQTsnp_285.png}
\caption{cluster plot for AtMSQTsnp 285.} \label{flAtMSQTsnp285}
\end{figure}

\begin{figure}[H]
\includegraphics[width=0.5\textwidth]{figures/cluster_plot_AtMSQTsnp_286.png}
\caption{cluster plot for AtMSQTsnp 286.} \label{flAtMSQTsnp286}
\end{figure}

\begin{figure}[H]
\includegraphics[width=0.5\textwidth]{figures/cluster_plot_AtMSQTsnp_288.png}
\caption{cluster plot for AtMSQTsnp 288.} \label{flAtMSQTsnp288}
\end{figure}

\begin{figure}[H]
\includegraphics[width=0.5\textwidth]{figures/cluster_plot_AtMSQTsnp_292.png}
\caption{cluster plot for AtMSQTsnp 292.} \label{flAtMSQTsnp292}
\end{figure}

\begin{figure}[H]
\includegraphics[width=0.5\textwidth]{figures/cluster_plot_AtMSQTsnp_294.png}
\caption{cluster plot for AtMSQTsnp 294.} \label{flAtMSQTsnp294}
\end{figure}

\begin{figure}[H]
\includegraphics[width=0.5\textwidth]{figures/cluster_plot_AtMSQTsnp_300.png}
\caption{cluster plot for AtMSQTsnp 300.} \label{flAtMSQTsnp300}
\end{figure}

\begin{figure}[H]
\includegraphics[width=0.5\textwidth]{figures/cluster_plot_AtMSQTsnp_304.png}
\caption{cluster plot for AtMSQTsnp 304.} \label{flAtMSQTsnp304}
\end{figure}

\begin{figure}[H]
\includegraphics[width=0.5\textwidth]{figures/cluster_plot_AtMSQTsnp_306.png}
\caption{cluster plot for AtMSQTsnp 306.} \label{flAtMSQTsnp306}
\end{figure}

\begin{figure}[H]
\includegraphics[width=0.5\textwidth]{figures/cluster_plot_AtMSQTsnp_307.png}
\caption{cluster plot for AtMSQTsnp 307.} \label{flAtMSQTsnp307}
\end{figure}

\begin{figure}[H]
\includegraphics[width=0.5\textwidth]{figures/cluster_plot_AtMSQTsnp_310.png}
\caption{cluster plot for AtMSQTsnp 310.} \label{flAtMSQTsnp310}
\end{figure}

\begin{figure}[H]
\includegraphics[width=0.5\textwidth]{figures/cluster_plot_AtMSQTsnp_312.png}
\caption{cluster plot for AtMSQTsnp 312.} \label{flAtMSQTsnp312}
\end{figure}

\begin{figure}[H]
\includegraphics[width=0.5\textwidth]{figures/cluster_plot_AtMSQTsnp_315.png}
\caption{cluster plot for AtMSQTsnp 315.} \label{flAtMSQTsnp315}
\end{figure}

\begin{figure}[H]
\includegraphics[width=0.5\textwidth]{figures/cluster_plot_AtMSQTsnp_321.png}
\caption{cluster plot for AtMSQTsnp 321.} \label{flAtMSQTsnp321}
\end{figure}

\begin{figure}[H]
\includegraphics[width=0.5\textwidth]{figures/cluster_plot_AtMSQTsnp_323.png}
\caption{cluster plot for AtMSQTsnp 323.} \label{flAtMSQTsnp323}
\end{figure}

\begin{figure}[H]
\includegraphics[width=0.5\textwidth]{figures/cluster_plot_AtMSQTsnp_325.png}
\caption{cluster plot for AtMSQTsnp 325.} \label{flAtMSQTsnp325}
\end{figure}

\begin{figure}[H]
\includegraphics[width=0.5\textwidth]{figures/cluster_plot_AtMSQTsnp_327.png}
\caption{cluster plot for AtMSQTsnp 327.} \label{flAtMSQTsnp327}
\end{figure}

\begin{figure}[H]
\includegraphics[width=0.5\textwidth]{figures/cluster_plot_AtMSQTsnp_331.png}
\caption{cluster plot for AtMSQTsnp 331.} \label{flAtMSQTsnp331}
\end{figure}

\begin{figure}[H]
\includegraphics[width=0.5\textwidth]{figures/cluster_plot_AtMSQTsnp_334.png}
\caption{cluster plot for AtMSQTsnp 334.} \label{flAtMSQTsnp334}
\end{figure}

\begin{figure}[H]
\includegraphics[width=0.5\textwidth]{figures/cluster_plot_AtMSQTsnp_343.png}
\caption{cluster plot for AtMSQTsnp 343.} \label{flAtMSQTsnp343}
\end{figure}

\begin{figure}[H]
\includegraphics[width=0.5\textwidth]{figures/cluster_plot_AtMSQTsnp_350.png}
\caption{cluster plot for AtMSQTsnp 350.} \label{flAtMSQTsnp350}
\end{figure}

\begin{figure}[H]
\includegraphics[width=0.5\textwidth]{figures/cluster_plot_AtMSQTsnp_351.png}
\caption{cluster plot for AtMSQTsnp 351.} \label{flAtMSQTsnp351}
\end{figure}

\begin{figure}[H]
\includegraphics[width=0.5\textwidth]{figures/cluster_plot_AtMSQTsnp_355.png}
\caption{cluster plot for AtMSQTsnp 355.} \label{flAtMSQTsnp355}
\end{figure}

\begin{figure}[H]
\includegraphics[width=0.5\textwidth]{figures/cluster_plot_AtMSQTsnp_358.png}
\caption{cluster plot for AtMSQTsnp 358.} \label{flAtMSQTsnp358}
\end{figure}

\begin{figure}[H]
\includegraphics[width=0.5\textwidth]{figures/cluster_plot_AtMSQTsnp_359.png}
\caption{cluster plot for AtMSQTsnp 359.} \label{flAtMSQTsnp359}
\end{figure}

\begin{figure}[H]
\includegraphics[width=0.5\textwidth]{figures/cluster_plot_AtMSQTsnp_360.png}
\caption{cluster plot for AtMSQTsnp 360.} \label{flAtMSQTsnp360}
\end{figure}

\begin{figure}[H]
\includegraphics[width=0.5\textwidth]{figures/cluster_plot_AtMSQTsnp_361.png}
\caption{cluster plot for AtMSQTsnp 361.} \label{flAtMSQTsnp361}
\end{figure}

\begin{figure}[H]
\includegraphics[width=0.5\textwidth]{figures/cluster_plot_AtMSQTsnp_368.png}
\caption{cluster plot for AtMSQTsnp 368.} \label{flAtMSQTsnp368}
\end{figure}

\begin{figure}[H]
\includegraphics[width=0.5\textwidth]{figures/cluster_plot_AtMSQTsnp_370.png}
\caption{cluster plot for AtMSQTsnp 370.} \label{flAtMSQTsnp370}
\end{figure}

\begin{figure}[H]
\includegraphics[width=0.5\textwidth]{figures/cluster_plot_AtMSQTsnp_372.png}
\caption{cluster plot for AtMSQTsnp 372.} \label{flAtMSQTsnp372}
\end{figure}

\begin{figure}[H]
\includegraphics[width=0.5\textwidth]{figures/cluster_plot_AtMSQTsnp_373.png}
\caption{cluster plot for AtMSQTsnp 373.} \label{flAtMSQTsnp373}
\end{figure}

\begin{figure}[H]
\includegraphics[width=0.5\textwidth]{figures/cluster_plot_AtMSQTsnp_378.png}
\caption{cluster plot for AtMSQTsnp 378.} \label{flAtMSQTsnp378}
\end{figure}

\begin{figure}[H]
\includegraphics[width=0.5\textwidth]{figures/cluster_plot_AtMSQTsnp_379.png}
\caption{cluster plot for AtMSQTsnp 379.} \label{flAtMSQTsnp379}
\end{figure}

\begin{figure}[H]
\includegraphics[width=0.5\textwidth]{figures/cluster_plot_AtMSQTsnp_388.png}
\caption{cluster plot for AtMSQTsnp 388.} \label{flAtMSQTsnp388}
\end{figure}

\begin{figure}[H]
\includegraphics[width=0.5\textwidth]{figures/cluster_plot_AtMSQTsnp_390.png}
\caption{cluster plot for AtMSQTsnp 390.} \label{flAtMSQTsnp390}
\end{figure}

\begin{figure}[H]
\includegraphics[width=0.5\textwidth]{figures/cluster_plot_AtMSQTsnp_392.png}
\caption{cluster plot for AtMSQTsnp 392.} \label{flAtMSQTsnp392}
\end{figure}

\begin{figure}[H]
\includegraphics[width=0.5\textwidth]{figures/cluster_plot_AtMSQTsnp_394.png}
\caption{cluster plot for AtMSQTsnp 394.} \label{flAtMSQTsnp394}
\end{figure}

\begin{figure}[H]
\includegraphics[width=0.5\textwidth]{figures/cluster_plot_AtMSQTsnp_395.png}
\caption{cluster plot for AtMSQTsnp 395.} \label{flAtMSQTsnp395}
\end{figure}

\begin{figure}[H]
\includegraphics[width=0.5\textwidth]{figures/cluster_plot_AtMSQTsnp_397.png}
\caption{cluster plot for AtMSQTsnp 397.} \label{flAtMSQTsnp397}
\end{figure}

\begin{figure}[H]
\includegraphics[width=0.5\textwidth]{figures/cluster_plot_AtMSQTsnp_398.png}
\caption{cluster plot for AtMSQTsnp 398.} \label{flAtMSQTsnp398}
\end{figure}

\begin{figure}[H]
\includegraphics[width=0.5\textwidth]{figures/cluster_plot_AtMSQTsnp_399.png}
\caption{cluster plot for AtMSQTsnp 399.} \label{flAtMSQTsnp399}
\end{figure}

\begin{figure}[H]
\includegraphics[width=0.5\textwidth]{figures/cluster_plot_AtMSQTsnp_404.png}
\caption{cluster plot for AtMSQTsnp 404.} \label{flAtMSQTsnp404}
\end{figure}

\begin{figure}[H]
\includegraphics[width=0.5\textwidth]{figures/cluster_plot_AtMSQTsnp_406.png}
\caption{cluster plot for AtMSQTsnp 406.} \label{flAtMSQTsnp406}
\end{figure}

\begin{figure}[H]
\includegraphics[width=0.5\textwidth]{figures/cluster_plot_AtMSQTsnp_408.png}
\caption{cluster plot for AtMSQTsnp 408.} \label{flAtMSQTsnp408}
\end{figure}

\begin{figure}[H]
\includegraphics[width=0.5\textwidth]{figures/cluster_plot_AtMSQTsnp_409.png}
\caption{cluster plot for AtMSQTsnp 409.} \label{flAtMSQTsnp409}
\end{figure}

\begin{figure}[H]
\includegraphics[width=0.5\textwidth]{figures/cluster_plot_AtMSQTsnp_410.png}
\caption{cluster plot for AtMSQTsnp 410.} \label{flAtMSQTsnp410}
\end{figure}

\begin{figure}[H]
\includegraphics[width=0.5\textwidth]{figures/cluster_plot_AtMSQTsnp_412.png}
\caption{cluster plot for AtMSQTsnp 412.} \label{flAtMSQTsnp412}
\end{figure}

\begin{figure}[H]
\includegraphics[width=0.5\textwidth]{figures/cluster_plot_AtMSQTsnp_415.png}
\caption{cluster plot for AtMSQTsnp 415.} \label{flAtMSQTsnp415}
\end{figure}



\end{document}
